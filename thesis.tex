% This is the Reed College LaTeX thesis template. Most of the work 
% for the document class was done by Sam Noble (SN), as well as this
% template. Later comments etc. by Ben Salzberg (BTS). Additional
% restructuring and APA support by Jess Youngberg (JY).
% Your comments and suggestions are more than welcome; please email
% them to cus@reed.edu
%
% See http://web.reed.edu/cis/help/latex.html for help. There are a 
% great bunch of help pages there, with notes on
% getting started, bibtex, etc. Go there and read it if you're not
% already familiar with LaTeX.
%
% Any line that starts with a percent symbol is a comment. 
% They won't show up in the document, and are useful for notes 
% to yourself and explaining commands. 
% Commenting also removes a line from the document; 
% very handy for troubleshooting problems. -BTS

% As far as I know, this follows the requirements laid out in 
% the 2002-2003 Senior Handbook. Ask a librarian to check the 
% document before binding. -SN

%%
%% Preamble
%%
% \documentclass{<something>} must begin each LaTeX document
\documentclass[12pt,twoside]{reedthesis}
% Packages are extensions to the basic LaTeX functions. Whatever you
% want to typeset, there is probably a package out there for it.
% Chemistry (chemtex), screenplays, you name it.
% Check out CTAN to see: http://www.ctan.org/
%%
%based off of cryptomacros file from Adam Groce for cryptography spring 2020
%this file is a mess since it is added packages and macros tacked on from outside sources :(

\usepackage{amsthm} %theorem environments etc
%\usepackage{fullpage} % reducing borders
%\usepackage{xspace}
\usepackage{amssymb} % \mathbb
\usepackage{amsmath}
\usepackage{amsfonts}
\usepackage[utf8]{inputenc}
%\usepackage{hyperref}
\usepackage{graphicx}
\usepackage[vlined]{algorithm2e}
\RestyleAlgo{boxruled}


\theoremstyle{remark}
\newtheorem*{rmk}{Remark}

\theoremstyle{plain}
\newtheorem{theorem}{Theorem}[chapter]
\newtheorem{corollary}[theorem]{Corollary}
\newtheorem{lemma}[theorem]{Lemma}
\newtheorem{prop}[theorem]{Proposition}

\theoremstyle{definition}
\newtheorem{dfn}[theorem]{Definition}

\setcounter{equation}{0}


\newcommand{\adv}{\mathcal{A}}
\newcommand{\reals}{\mathbb{R}}
\newcommand{\gen}{\ensuremath{{\sf Gen}}\xspace}
\newcommand{\enc}{\ensuremath{{\sf Enc}}\xspace}
\newcommand{\dec}{\ensuremath{{\sf Dec}}\xspace}
\newcommand{\negl}{\ensuremath{{\sf negl}}\xspace}
\newcommand{\EXP}{\mathbb{E}}
\newcommand{\bidders}{\textbf{b}}
\newcommand{\valuations}{\textbf{v}}
\newcommand{\strategies}{\textbf{s}}
\newcommand{\types}{\textbf{T}}
\newcommand{\actions}{\textbf{a}}
\newcommand{\OPT}{\text{OPT}}
\newcommand{\F}{\mathcal{F}}

\usepackage[procnames]{listings}
\usepackage{color}
\usepackage{xcolor}

\lstset{language=Python, 
	basicstyle=\ttfamily\small, 
	keywordstyle=\color{keywords},
	commentstyle=\color{comments},
	stringstyle=\color{red},
	showstringspaces=false,
	identifierstyle=\color{green},
	procnamekeys={def,class}}

%\usepackage{systeme}

%\usepackage[textheight=8.5in, margin=1.6in]{geometry}
\usepackage[total={6in,8in}]{geometry}

\usepackage{graphicx}

% Tikz stuff: This comes from Dave, hence it being added on at end
\usepackage{tikz}
\usetikzlibrary{shapes.misc, positioning}
\usepackage{pgfplots}
\pgfplotsset{compat=1.13}
%\usepgfplotslibrary{fillbetween}
%\usetikzlibrary{cd}
%\usetikzlibrary{graphs,graphs.standard}
%\usetikzlibrary{decorations.pathmorphing}    
%\usetikzlibrary{decorations.pathreplacing}
%\usetikzlibrary{arrows.meta,calc}
%\usetikzlibrary{bending}
\usetikzlibrary{decorations.markings,shapes.geometric}
\tikzset{->-/.style={decoration={markings, mark=at position #1 with
			{\arrow{>}}},postaction={decorate}}}
\tikzset{-|-/.style={decoration={markings, mark=at position #1 with
			{\arrow{stealth}}},postaction={decorate}}}
\tikzset{movearrow/.style 2 args ={
		decoration={markings,
			mark= at position {#1} with {\arrow{#2}} ,
		},
		postaction={decorate}
	}
}
\tikzset{<--/.style={decoration={markings, mark=at position #1 with
			{\arrow{<}}},postaction={decorate}}}
\tikzstyle{ball} = [circle,shading=ball, ball color=black,
minimum size=1mm,inner sep=1.3pt]
\tikzstyle{bball} = [circle,shading=ball, ball color=blue,
minimum size=1mm,inner sep=1.3pt]
\tikzstyle{miniball} = [circle,shading=ball, ball color=black,
minimum size=1mm,inner sep=0.5pt]
\tikzstyle{redminiball} = [circle,shading=ball, ball color=red,
minimum size=1mm,inner sep=0.5pt]
\tikzstyle{mminiball} = [circle,shading=ball, ball color=black,
minimum size=0.6mm,inner sep=0.1pt]
\tikzstyle{mycircle} = [draw,circle, minimum size=1mm,inner sep=1.3pt]

\usepackage{needspace}  % <-- to prevent bad pagebreak over environment titles
% Use:  \Needspace{3\baselineskip} before opening an environment

\usepackage{enumitem}
%\setlist[enumerate]{leftmargin=*}
%\setlist[enumerate]{itemsep=2pt,label={(\alph*)}}

%\allowdisplaybreaks
%\usepackage[colorlinks=true,citecolor=black,linkcolor=black,urlcolor=blue]{hyperref}
%\usepackage{algorithm}
%\usepackage[noend]{algpseudocode}

\usepackage{outlines}



	
	



\usepackage{graphicx,latexsym} 
\usepackage{amssymb,amsthm,amsmath}
\usepackage{longtable,booktabs,setspace} 
\usepackage{chemarr} %% Useful for one reaction arrow, useless if you're not a chem major
\usepackage[hyphens]{url}
\urlstyle{same}
\usepackage{rotating}
\usepackage{natbib}
\usepackage{changepage}
\usepackage[vlined]{algorithm2e}
\usepackage{amsthm}
\usepackage [english]{babel}
\usepackage [autostyle, english = american]{csquotes}
\MakeOuterQuote{"}

\graphicspath{./diagrams/}
% Comment out the natbib line above and uncomment the following two lines to use the new 
% biblatex-chicago style, for Chicago A. Also make some changes at the end where the 
% bibliography is included. 
%\usepackage{biblatex-chicago}
%\bibliography{thesis}

% \usepackage{times} % other fonts are available like times, bookman, charter, palatino

\title{Simulating the Price of Anarchy in Auctions}
\author{Robert S. Irvin}
% The month and year that you submit your FINAL draft TO THE LIBRARY (May or December)
\date{May 2020}
\division{Economics and Mathematics}
\advisor{Jeffery Parker}
%If you have two advisors for some reason, you can use the following
\altadvisor{David Perkinson}
%%% Remember to use the correct department!
\department{Economics-Mathematics}
% if you're writing a thesis in an interdisciplinary major,
% uncomment the line below and change the text as appropriate.
% check the Senior Handbook if unsure.
\thedivisionof{The Established Interdisciplinary Committee for}
% if you want the approval page to say "Approved for the Committee",
% uncomment the next line
\approvedforthe{Committee for}

\setlength{\parskip}{0pt}
%%
%% End Preamble
%%
%% The fun begins:
\begin{document}

  \maketitle
  \frontmatter % this stuff will be roman-numbered
  \pagestyle{empty} % this removes page numbers from the frontmatter

% Acknowledgements (Acceptable American spelling) are optional
% So are Acknowledgments (proper English spelling)
    \chapter*{Acknowledgments}
	The cat, Pinoe

% The preface is optional
% To remove it, comment it out or delete it.
%    \chapter*{Preface}
%	"For if we could suppose a great multitude of men to consent in the observation of justice and other laws of nature without a common power to keep them in awe, we might as well suppose all mankind to do the same; and then there neither would be, nor need to be, any civil government or commonwealth at all, because there would be peace without subjection." -Thomas Hobbes (Leviathan, chapter XVII)
	
	%TODO find a better leviathan quote
	
	

%    \chapter*{List of Abbreviations}
%
%	\begin{table}[h]
%	\centering % You could remove this to move table to the left
%	\begin{tabular}{ll}
%		\textbf{AI}  	&  Artificial Intelligence\\
%		\textbf{MAS}  	&  Multi-Agent System\\
%		\textbf{ML}     &  Machine Learning\\
%		\textbf{POA}    &  Price of Anarchy
%	\end{tabular}
%	\end{table}
	

    \tableofcontents
% if you want a list of tables, optional
%    \listoftables
%% if you want a list of figures, also optional
%    \listoffigures

% The abstract is not required if you're writing a creative thesis (but aren't they all?)
% If your abstract is longer than a page, there may be a formatting issue.
    \chapter*{Abstract}
    
    The purpose of this thesis is to demonstrate via simulation the efficiency guarantees of the price of anarchy in first-price, single-item auctions. An agent based model is used where each agents strategy can be updated to demonstrate the efficiency properties of the system under this strategy. 
    
    The simulations find that agents using known Bayes-Nash equilibria strategies converge to a price of anarchy greater then the lower bound given by theory. Next, we see that agents who use an arbitrary bidding strategy still achieve high efficiency in both symmetric and asymmetric auctions despite their lack of intelligence, and that as the number of bidders increases these auctions approaches full efficiency. Finally, we see that no-regret learning agents converge to outcomes both better than the efficiency guarantees that theory would suggest, and better than the arbitrary bidders.  
    
    The results from this experiment both demonstrate the correctness of the price of anarchy theory and give high confidence in efficiency of first-price single payer auctions. While even arbitrary agents perform quite well, we see that agents who are capable of learning and reflecting human behavior converge to highly efficient outcomes.  
	
%	\chapter*{Dedication}
%	You can have a dedication here if you wish.

  \mainmatter % here the regular arabic numbering starts
  \pagestyle{fancyplain} % turns page numbering back on

%The \introduction command is provided as a convenience.
%if you want special chapter formatting, you'll probably want to avoid using it altogether

    \chapter*{Introduction}
         \addcontentsline{toc}{chapter}{Introduction}
	\chaptermark{Introduction}
	\markboth{Introduction}{Introduction}
	% The three lines above are to make sure that the headers are right, that the intro gets included in the table of contents, and that it doesn't get numbered 1 so that chapter one is 1.

% Double spacing: if you want to double space, or one and a half 
% space, uncomment one of the following lines. You can go back to 
% single spacing with the \singlespacing command.
% \onehalfspacing
\doublespacing

When people act rationally, society can suffer. Economists often talk about the tragedy of the commons, a scenario where people follow their incentives to over utilize a public good and ruin it for society at large. In the tragedy of the commons, each agent's rational behavior is what leads the system to converge to overuse, wasted resources, and utility lower in the long run than if they had been able to preserve the resource. While the idea of the tragedy of the commons is about how incentives can lead to the overuse of public resources, there is a larger problem at the heart of it: how and why is it that when people behave strategically it often leads to worse outcomes for both the individual and society as a whole? 

The fact that selfish behavior leads to a worse outcome in the tragedy of the commons is a product of the system it takes place in. Because the incentives of this game line up with individualistic choices that are bad for society, rational agents will always choose to over utilize the public good. It would only be if some outside force came in and made people behave correctly that these agents could maintain behaviors that would lead to the best outcome for society in the long run. Socially optimal choice for all agents is not an equilibrium when agents behave rationally in this case, but we can still imagine what it would be like if agents were forced to make socially responsible choices by some external force. This is an economic dream, to have some {\em deus economica} control the actions of every agent to maximize some empirically derived social welfare function. Every action preplanned and every util maximized for the good of us all. In lieu of an economic god ruling us, maybe some benevolent dictator could be set up instead? 

Even if a benevolent dictator can be found, who is to say that they will be able to calculate the exact optimal solutions for large scale problems? Imagine a nation's government trying to calculate the socially optimal number of shoes to manufacture each year for millions of people each with different shoe sizes and preferences? It seems doubtful that any government would be capable of doing this as there is too much data to process and too many changing circumstances to react quickly to societal need. Rather, most countries leave shoe manufacturing to a free market which has firms that can pop up and respond to the demands of customers at will. That is not to say that the free market is perfect for shoes, we have no way of knowing if its allocation of resources end up being the best possible. In fact, it is highly doubtful that it is. There is only one socially optimal outcome (which presumably our omnipotent god of economics could calculate) but there are infinite number of outcomes which are worse than optimal. Our question is one of trying to understand what makes some markets or systems behave in a way that is good for society and some terrible for society when everyone is behaving selfishly, or, how can we describe the effect of selfish behavior on a system. One way of approaching an answer to this question is instead of trying to understand how agents behave in the worst case (from a social welfare perspective) for a given system or game. That is, we want to put bounds systems worst possible social cost when people are behaving rationally. This is called the {\em price of anarchy}, the worst possible price paid in terms of social welfare  at the equilibria for rational agents within that system. 

The idea of equilibria in a system is also problematic. How are we to presume that agents converge to an equilibria? Are they able to easily calculate it and decide to go there to begin with? No. We should be concerned with the set of outcomes or equilibria that happens not only for fully rational agents with infinite computational power, but also with agents who learn how to behave as they participate, who dynamically react to the changes in strategy by the other players. 

This thesis is about simulating the price to society for anarchic behavior within a system (game) who learn to play as they go. Computer scientists and economists have been proving the bounds of this price for various systems both at the fully rational equilibria, and at the equilibria that certain learning agents converge to. We create a simulation to demonstrate this bound in one of the most simple markets, a sealed bid first-price, single-item auction. This is an auction where everyone bids simultaneously, one-shot on a single item and the highest bidder gets the item. Thus, by constructing a simulation of an auction in this way, we can be more sure that when this is used in real life that these auctions are producing outcomes that are at least as efficient as our lower bound, and we no longer have to worry about finding a benevolent dictator.
	 
	
\chapter{Auctions and Computational Economics}
	Within the last three decades, computational economics has been on the rise. This branch of economic research encompasses two major ideas. One is that the increasing power of computers can help solve and understand classical economic problems through increasingly more complex simulations and numerical analysis. Two, that the mathematical methods developed in the field of theoretical computer science can be used to gain better understanding of existing models in terms of their algorithmic and computational complexity properties. We aim to take elements from both of these frameworks to better understand the social welfare of auctions at equilibrium. 

\section{Auctions, Equilibria, and Anarchy}
It is perhaps obvious why economists would be interested in studying auctions. Auctions are one of the most basic market structures that have roots going back to at least the ancient Greeks and still exist today in places such as art auctions and Ebay \citep{Mochon2015}. In the past 15 years, economists have been joined by computer scientists who are increasingly interested in the strategic interactions of agents within this setting. Christos Papadimitriou said in a 2015 lecture at the Simons Institute that it was the advent of the internet, an artifact out of computer scientists control, that turned theoretical computer science into a "physical science." Now computer scientists had to "approach the internet with the same humility that economists approach the market..." He went on to say that "it also turned us [computer science] into a social science. It was obviously about people and incentives. Without understanding this, you cannot understand the internet"\citep{Papadimitriou2015}. It is within this framework that he says computer scientists first began to study auctions as they existed on the internet such as Ebay and Google's sponsored search auctions. Now, computer scientists are moving beyond the internet, taking the mathematical tools of their discipline and applying them as a lens to understand and explain the world. This thesis is looking at the field algorithmic game theory, the study of the algorithms and complexity of strategic interactions. For economists, this can be thought of as a new toolbox for unpacking and understanding the models and structures that already dominate the field. For example, in the case of a Walrasian auctioneer who calculates the clearing prices of a combinatorial auction, their problem was shown to be NP-complete, a complexity class usually called "intractable" due to the time it takes to solve these problems (the best algorithms here are generally guess and check, which gets out of hand for large inputs i.e. possible combinations)\citep{Papadimitriou2015}\footnote{NP is the class of problems that a given solution can be checked in polynomial time, i.e. $O(n^k)$ operations where $n$ is the size of the input and $k$ is any positive integer. NP-complete means that all problems in NP reduce to solving this problem. This is when you will usually hear "taking more time to compute than the age of the universe" etc...}. Using the lens of computational complexity, analysis of how many operations or how much memory a problem must take to solve, is one way of assessing what assumptions we are making about the computational power of our rational agents.

\subsection{The Price of Anarchy}
One of the major ideas to come out of algorithmic game theory is the \textit{price of anarchy} (POA), a mathematical way of showing the difference between the social welfare in the optimal case and in the worst case equilibrium for a game. This term was first coined by two computer scientist Elias Koutsoupias and Christos Papadimitriou, who were using the price of anarchy to understand network games and more generally were looking at how this concept could be used to understand behavior on the internet \citep*{Koutsoupias1999, Papadimitriou2001}. More formally, the price of anarchy for a game is the ratio of the minimum equilibrium social welfare in a game over the best possible social welfare of the game (The actual formal definition is given in definition \ref{eq:POA}). An illustrative example of the price of anarchy from \citet[15-16]{Roughgarden2016} are selfish routing games played on graphs such as the one pictured below in figure \ref{braess}.
 

\begin{figure}[h!]
	\begin{tikzpicture}[scale=3.0]
	\node[circle,draw,inner sep=3pt, minimum size=15pt] at (0,0) (left) {$s$};
	\node[circle,draw,inner sep=3pt, minimum size=15pt] at (1,1) (top) {};
	\node[circle,draw,inner sep=3pt, minimum size=15pt] at (1,-1) (bottom) {};
	\node[circle,draw,inner sep=3pt, minimum size=15pt] at (2,0) (right) {$t$};
	\draw[->] (left)--(top) node[midway,above,rotate=45] {$c(x)=x$};
	\draw[->] (top)--(right) node[midway,above,rotate=-45] {$c(x)=1$};
	\draw[->] (top)--(bottom) node[midway,above,rotate=-90] {$c(x)=0$};
	\draw[->] (left)--(bottom) node[midway,below,rotate=-45] {$c(x)=1$};
	\draw[->] (bottom)--(right) node[midway,below,rotate=45] {$c(x)=1$};
	\end{tikzpicture}
	\centering
	\caption{Routing game from s to t. Strategic interaction will make everyone worse off.}
	\label{braess}
\end{figure}

In this game, each player chooses which edges to take on the directed graph from the source $s$ to reach the terminal $t$ and tries to do it with the least cost, total weight of edges taken, as possible. The cost of taking these edges it is a function of the proportion of players who take that edge in the game. Some of them do not depend on how many players take it, but always cost a constant $1$ or $0$ where others are exactly proportional to how many players take it with a possible range $(0,1]$.
For example if $50 \%$ of the players take that edge then $x=0.5$. This can be seen as analogous to traffic when driving a car, the more people take a road, the slower the traffic goes and the longer it takes to get somewhere. Knowing this, each player try to choose which path to take to get to their destination as quickly as possible. As you can quickly verify, the best solution for society, minimizing the total driving time for all, is when half of the drivers take the top route, and half of the drivers take the bottom route costing in total 1.5 for each driver. 

\begin{figure}[h!]
	\centering
	\begin{tikzpicture}[scale=3.0]
	\node[circle,draw,inner sep=3pt, minimum size=15pt] at (0,0) (left) {$s$};
	\node[circle,draw,inner sep=3pt, minimum size=15pt] at (1,1) (top) {};
	\node[circle,draw,inner sep=3pt, minimum size=15pt] at (1,-1) (bottom) {};
	\node[circle,draw,inner sep=3pt, minimum size=15pt] at (2,0) (right) {$t$};
	\draw[->,red,thick] (left)--(top) node[midway,above,rotate=45] {$c(x)=x$};
	\draw[->,red,thick] (top)--(right) node[midway,above,rotate=-45] {$c(x)=1$};
	\draw[->] (top)--(bottom) node[midway,above,rotate=-90] {$c(x)=0$};
	\draw[->,cyan,thick] (left)--(bottom) node[midway,below,rotate=-45] {$c(x)=1$};
	\draw[->,cyan,thick] (bottom)--(right) node[midway,below,rotate=45] {$c(x)=1$};
	\end{tikzpicture}
	\caption{Socially optimal Routing}
	\label{braess2}
\end{figure}

However, this is not an equilibrium. The drivers taking the top route have incentive to instead take the middle edge to try and lower their total time traveled. In fact, the Nash equilibrium will end with all of the drivers taking the top x, going through the middle, and then the bottom x. Only then will no players have reason to deviate. These actions to individually and strategically try to decrease their cost end up producing a worse outcome for all players and society as a whole (the social welfare)\footnote{A careful observer will note that if the edge with cost 0 were not there, this would not be a problem. This is called Braess's Paradox where having this extra edge counter intuitively leads to worse outcomes \citep{Braess1968}.}.  

\begin{figure}[h!]
	\centering
	\begin{tikzpicture}[scale=3.0]
	\node[circle,draw,inner sep=3pt, minimum size=15pt] at (0,0) (left) {$s$};
	\node[circle,draw,inner sep=3pt, minimum size=15pt] at (1,1) (top) {};
	\node[circle,draw,inner sep=3pt, minimum size=15pt] at (1,-1) (bottom) {};
	\node[circle,draw,inner sep=3pt, minimum size=15pt] at (2,0) (right) {$t$};
	\draw[->,blue,thick] (left)--(top) node[midway,above,rotate=45] {$c(x)=x$};
	\draw[->] (top)--(right) node[midway,above,rotate=-45] {$c(x)=1$};
	\draw[->,blue,thick] (top)--(bottom) node[midway,above,rotate=-90] {$c(x)=0$};
	\draw[->] (left)--(bottom) node[midway,below,rotate=-45] {$c(x)=1$};
	\draw[->,blue,thick] (bottom)--(right) node[midway,below,rotate=45] {$c(x)=1$};
	\end{tikzpicture}
	\caption{Route under strategic interaction}
	\label{braess3}
\end{figure}

Now, it will take time 2 for each player to reach the terminal. Thus, under strategic interaction we see that the equilibrium is sub-optimal for society, and in this case all players as well. In this case, the price of anarchy can be computed as $$POA = 3/4$$ \citep{Roughgarden2017}. For this network, there is only one (Nash) equilibrium for selfish routing games, however, price of anarchy bounds can be found for the class of game as a whole regardless of its individual construction. These bounds tell us how much worse then optimal we can expect a system at equilibrium to behave in the worst case. The price of anarchy bound for these selfish routing games with affine cost functions is know to be $3/4$, meaning that our example is as bad as the price of anarchy could get for any network \citep{Roughgarden2007}.

\subsection{Auctions}
This framework of the price of anarchy is now being applied to understand auctions by researchers in the algorithmic game theory field. Where for selfish routing games, the price of anarchy has been shown to have the tight bound above, the price of anarchy for single price auctions only has approximate upper and lower bounds set on it, and is not known at all for double auctions. Before exploring this further, we should set up the mathematical framework of auctions that is necessary to compute such bounds. 

An \textit{auction} is a market mechanism, operating under specific rules that determines to whom one or more items will be awarded and at what price. For a bidder in an auction, the \textit{value} $v_i$ is how much the bidder values the item. This is sometimes called a private valuation as this value is generally unknown by the other participants in the auction. The \textit{bid}, $b_i$, is the offer that bidder $i$ submits for an item. \textit{Sincere bidding} is when $v_i = b_i$, \textit{underbidding} is when $v_i > b_i$, and \textit{overbidding} is when $v_i < b_i$. The highest bid made by any bidder is denoted $b^*$, and is the winning bid (if multiple bids equal the winning bid, then some tie-breaking rule must be used). The \textit{selling price}, $p^*$ is the final price that the bidder actually pays for the item (which depending on the auction type need not equal $b^*$). Under the \textit{first-price} rule, the bid submitted by the winner is equal to the selling price. Before the auction begins, each bidder knows their personal or private value for the item. An auction consists of a set of bidders, $I = (1,2, ...,N)$ and a seller. After the auction, the bidder $i$ wins the item if their bid is higher than the bid placed by any other bidder $k$, $b_i > max_{k \neq b_k}$. In a single unit auction, the \textit{income of the bidder} $i$ is equal to their value of the item: $$ \Gamma_i^* = v_i$$ and the \textit{surplus of the bidder} $i$ is equal to the difference between income and price paid $$ \Pi_i^* = \Gamma_i^* - p^*.$$ If a bid placed by a bidder is less than the winning bid, they do not win anything and their income and surplus are both zero. The {\em seller's revenue} in a single unit auction is equal to the price paid by the winning bidder $$ R^* = p^*.$$ 

There are multiple different pricing rules in an auction that determine who gets allocated the item. In a \textit{first-price} auction the winning bidder pays the amount of their bid, which is the highest bid of the auction: $p^* = b^*$. Also called \textit{pay-what-you-bid} (PWYB). In a \textit{second-price} auction, the winning bidder pays an amount that is equal to the second highest bid for the awarded item \citep{Vickrey1961} \footnote{The second price, sealed bid auction is also known as a Vickrey auction after the economist who invented it.}. People tend to bid lower than their private valuation in first price auctions since if they bid that value, their profit is zero regardless of if they win. Thus, the first price rule the item could be awarded to someone who values the item less than other bidders. Second price auctions encourages bidder to bid their true values (as they will gain positive profits if they win no matter the second highest bid). This encourages an efficient allocation of items \citep{Mochon2015}. For the moment we will confine ourselves to first price auctions as this is where most of the strong results in price of anarchy analysis of auctions currently are.

Before discussing the known bounds on the price of anarchy for first-price, single-item auctions, it is worth understanding how the bids might lead to a non-optimal outcome and what we mean by that. Due to only knowing their own valuation of the good, each bidder must act under uncertainty as to how much they should bid to beat out the unknown valuations of the other bidders. However, if they bid their exact private valuation then they will get a surplus of zero, or zero utility. In order for them to get some utility for the item they must be paying less then the exact amount they value the item. How much each bidder should bid less than their valuation, or {\em shade}, their bid is determined by how much they think that the other party values the item. To capture this interaction auctions are represented as Bayesian games where each bidder is drawing their bids from distributions known to the other player (that need not be the same). If one player knows that the other is drawing from a distribution with a smaller mean than they are (i.e. probably doesn't value the item as much), the Bayes-Nash equilibrium will have them shade their bid less and this other person will shade their bid more. This can lead to the person who values the item less winning the auction and creating less social welfare (the summed surplus of all bidders and the seller). 

Syrgkanis and Tardos proved in 2013 that the lower bound of the price of anarchy in first-price, single-item auctions is at least $1 - \frac{1}{e} \approx 0.63$. The exact upper bound is unknown, but it has been proven that it can be no better than $0.87$ \citep{Hartline2015}. This bound is true regardless of how many bidders there are or what distributions they are drawing their bids from so for any first-price single payer auction at equilibrium, we can say that it must be performing at least $63\%$ as well as it could in the best case scenario. 

This moves closer to answering the question for what is the price of anarchy at equilibrium, but these results do not pay attention to how the players arrive at these equilibria. In real world applications, we expect that players might play repeatedly in the same auction and learn as they play rather than come in with pre-computed strategies. This is especially true for when computing the equilibrium is computationally hard and the stakes of each individual auction is small. Given these observations, it is natural to ask questions about how the efficiency results carry over to adaptive game environments. The model for learning agents that is commonly used in the field is {\em no-regret learning}. An algorithm for a player satisfies the no-regret condition if, in the limit as the number of times the game is played goes to infinity, the average reward of the algorithm is at least as good as the average reward for the best fixed action in hindsight (assuming the sequence of actions for the other players remains unchanged) \footnote{That is to say that the algorithm will converge to having a loss no worse than any fixed strategy we would have rather picked in hindsight as the limit goes to infinity (see definition. A more precise definition, example algorithms, and uses will be shown in chapter 2 to clarify what "regret" is and how this converges to zero (see definition \ref{dfn:noregret}).}. If each player  incorporates this kind of learning algorithm, then it has been shown that these can converge to a larger class of equilibrium called coarse correlated equilibrium where each player conditions their response on the expected action of the other player \citep{Blum2007}. Luckily, the previous theorem has been extended so that we know that the price of anarchy for the set of coarse correlated equilibria of first price auctions are also at least 0.63 \citep{Roughgarden2017}. With all of this set up, we now state our goal: to simulate no-regret learning algorithms for agents in a first-price, single-payer auction to see how well the price of anarchy holds. 

	
\section{Simulated Agents and Simulated Economies}
The history of computers in economics, as outlined in \citet{Backhouse2016}, goes all the way back to general purpose computers being invented in the 1940's. Wassily Leontif used a computer to invert a 39 x 39 matrix to help solve his input output model. Since then, computers use in economics has exploded. With computers, economists are able to solve bigger matrices, do Monte-Carlo simulations, create multinomial probit models, and use full information maximum likelihood estimation. While one branch of computational economics is focused on creating stronger and stronger calculators to facilitate empirical research, another branch has focused on creating simulated economies that allow economists to construct a blended version of theory and research within a computer program. Within these simulated economies, theories can be coded into the simulation which, when run, can allow the researcher to conduct experiments that might not be practical to conduct in the real world. 

One kind of a simulation that can be run is called an agent-based model, a simulated system of autonomous decision makers (agents). These models are able to generate complex behavior even if only simple assumptions are made about the behavior of the coded agents. That is, these agents interacting with each other in complex ways are able to produce emergent phenomena in the macro structure of the system.

For example, in the late 1960's and early 1970's, Thomas Schilling created computer simulations to try and understand how and why self segregated neighborhoods formed. He coded a virtual environment where agents were given a simple preference, they are only happy if they are not the minority in their neighborhood and will keep moving otherwise \citep{Schelling1969}. This simple model can illustrate the main tenets of agent based modeling. The implementation we illustrate comes from Thomas Sargent and John Stachurski's 2019 lectures in quantitative economics \footnote{These lectures are a treasure trove of information on how to use python to construct economic models. The simulation in chapter 3 was built in part using them as a guide.}. First, we have agents who are representative of people in the real world. Their preferences to this respect are simple and easy to understand where they are only "happy" if half of their closest neighbors are the same as them. If they are not happy they will move somewhere else arbitrarily. These preferences can be represented as the short procedure, or algorithm, shown below where $S$ is just the space they live in.

\begin{adjustwidth}{1cm}{}
	\textbf{1.} Draw a random location in $S$ \\
	\textbf{2.} If happy at new location, move there\\
	\textbf{3.} Else, go to step 1
\end{adjustwidth}

In this case, we get to choose what that environment looks like and like Shelling we can just say that it is a one by one unit square and we can say that their neighbors are the ten closest people to them (in Euclidean distance) on that square. When you run this simulation with green and orange dots representing the types of people you get the following behavior cycling through each of our 250 agents with the above procedure until every agent is happy. 

\begin{figure}[h!]
	\centering
	\includegraphics[scale=0.5]{Figures/segregation_1}
	\caption{Schelling's Segregation Model: Cycle 1}
	\label{SSM1_ch1}
\end{figure}

\begin{figure}[h!]
	\centering
	\includegraphics[scale=0.5]{Figures/segregation_2}
	\caption{Schelling's Segregation Model: Cycle 2}
	\label{SSM2_ch1}
\end{figure}
\begin{figure}[h!]
	\centering
	\includegraphics[scale=0.5]{Figures/segregation_3}
	\caption{Schelling's Segregation Model: Cycle 3}
	\label{SSM3_ch1}
\end{figure}
\begin{figure}[h!]
	\centering
	\includegraphics[scale=0.5]{Figures/segregation_4}
	\caption{Schelling's Segregation Model: Cycle 4}
	\label{SSM4_ch1}
\end{figure}
\begin{figure}[h!]
	\centering
	\includegraphics[scale=0.5]{Figures/segregation_5}
	\caption{Schelling's Segregation Model: Cycle 5}
	\label{SSM5}
\end{figure}

As can be seen, in cycle 1 (Figure \ref{SSM1_ch1}), the agents are well distributed among each other, but as they move in cycles 2-5, they become progressively more segregated. After 5 cycles all agents are happy and the simulation terminates. With very few assumptions about the agents' preferences, we can see the resulting emergent behavior of segregated neighborhoods in the system as a whole. Not only that, but the agents naturally move to an equilibrium as they adjust their behavior to what their neighbors are doing \citep{Sargent2019}. 

Agent based modeling has been used to build and understand much more complicated systems then the example illustrated above. The Santa Fe Institute in New Mexico is one of the main proponents of agent based modeling releasing a manifesto supporting using it to understand the "complexity" of economics from the ground up \citep{Backhouse2016}. They built the Santa Fe Artificial Stock Market in the 1990's to try and simulate the behavior of agents on the stock market and how they adapt their adapting trading strategies effects the outcome of the market. This is one of the first examples of agents learning and adapting to their environments as part of the model \citep{LeBaron2002}. In this model agents used a genetic algorithm to adapt their trading strategy at each period by modifying a string (for example $00011100$) where each bit in the string told the agent to use a certain behavior or not. Those trading strategies that did well were coded to survive longer where the agents with worse strategies would randomly modify their own or take a more successful agents strategy \citep{Arthur1992}. This was supposed to be representative of the learning of traders on the stock market so that the insights taken from running these agents in simulation could be applied to learn something about the real world. 

The authors of Santa Fe Stock Market paper at the time suggested that this was one of many algorithms that could be used to stand in for human behavior saying that reinforcement learning or deep learning could also be used to stand in for human intelligence\footnote{Reinforcement learning is when an algorithm plays a game repeatedly and updates its beliefs about what actions will lead to the best payoff. Deep learning uses deep neural networks to try and estimate the best outcomes in a fashion similar to regression.}. More specifically they suggest that an appropriate algorithms can be chosen so long as they behave reasonably well like humans in that scenario, something like a Turing test (\cite{Arthur1991}). Recent research in the field as well as in the field of multi-agent systems, a similar branch of computer science suggest that this might not be possible in the general case with such simple (or not fully intelligent) algorithms. Because these simulations have agents competing in non-stationary environments that are changing from the perspective of any individual agent every time some other agent changes their behavior, the choice of algorithm dramatically changes how the system behaves (\cite{Rejeb2005}, \cite{Shoham2008}). Further, research that has compared the behavior of human agents in strategic settings to that of these algorithms have found that there is no general algorithm that best approximates human learning\citep{Tesfatsion2002}. From situation to situation, different algorithms more appropriately behave like humans. This is a problem only assuming that you want your agents to in some way represent human behavior, we might simply want our agents to represent the "rational choice" in any given situation. These algorithms aren't necessarily doing that either. As Holland and Miller say in their 1991 paper "Artificial Adaptive Agents in Economic Theory", "Usually there is only one way to be fully rational, but there are many ways to be less rational." The way they suggest to get around this is to try and build models that have robust behavior across algorithm choice.

The literature of game theory provides a nice solution to trying to approximate human behavior. The no-regret algorithms they use have the one simple property of doing better in the long run than doing any fixed strategy from the beginning. This is a simple learning requirement that means that this kind of algorithm should (hopefully) be more robust across representing human behavior. It also allows us to combine our simulated models and analytical models in a nice way as we can mathematically describe the processes of such algorithms behavior and we can code agents to use algorithms that have that behavior. This allows the learning process to be less of a black box and instead to be as simple and thus generalizable as possible.

Moving forward, this thesis aims to better understand the price of anarchy in auctions by simulating auctions using these no-regret learning algorithms and seeing how they behave compared to the socially optimal equilibrium. These simulations will try to answer two main questions, what equilibrium do these algorithms converge to (if they converge at all) and what is the calculated price of anarchy in these systems compared to the bounds that theory tells us should exist. Using this framework to explore auctions, we do this with the first-price, single-item auction as it is the best understood theoretically and is the simplest to code. 

\chapter{Price of Anarchy in First-Price Auctions}
This chapter lays out the mathematical framework for modeling auctions as Bayesian games of incomplete information, formally defines the price of anarchy in auctions, and shows the bounds on the price of anarchy for first price, single payer auctions. It then follows up on this framework of analysis to show how these bounds can be extended to agents competing in auctions using learning algorithms with a property called "no-regret" learning. This chapter primarily follows from the work of Tim Roughgarden, Vasilis Syrgkanis, and \'Eva Tardos who not only are all individually active in the field of algorithmic game theory and auctions, but also jointly authored the 2017 paper "The Price of Anarchy in Auctions" that is a survey of the entire topic. This chapter will be giving the important theorems and ideas that primarily come from this work and the individual work of these three authors, explaining them, demonstrating proofs where appropriate, and will reconstruct some of the results that we demonstrate with simulation.   

\section{Auctions as Bayesian Games}
Auctions are typically modeled as Bayesian games, also known as games with incomplete information. As you would expect, this is simply a type of game where the players don't know everything. While it is obvious why an auction consisting of the strategic interaction of bidders could be modeled as a game, how it should be modeled requires some thought. After all, while each bidder knows their own valuation of the item being sold, unless for some reason (and against their own interest) the other players announced their valuation of the item before the auction began our bidder will not know how the other players value the item. In fact, it is precisely this lack of information in sealed bid auctions that makes them interesting! If all players came into an auction knowing the valuation of other players, for first-price, single-item auctions assuming there was no tie, the bidder with the highest value would always win by bidding the valuation of the next highest bidder. Rather it is the uncertainty that players face about the valuation of other players that make things interesting as players must guess how much they should shade their bid (as again no player will ever bid above their valuation) based off of what they know about other players. 

What can we say that bidders know about the valuations of other bidders? Certainly they do not know nothing as we all have reasonable expectations about what some item is worth to others. No one will value a candy bar at a million dollars. But a collector of candy bars might value it at a higher value than an ordinary person who just want to eat the candy bar. For each bidder, we could say that this bidder has a probability distribution from which they are drawing their valuation from that is known to the other bidders. As an example we might expect normal people's valuations of candy bars to be a normal distribution centered at \$1, but the collector's value might be a Laplacian distribution (allowing for more black swan events) centered at \$5. This sort of strategic interaction where the players know the distributions which parameters of the game are drawn from are typically called games of incomplete information, or Bayesian games.
 
\subsection{Bayesian Games}
Bayesian games of incomplete information are games in which one or more of the players don't have "full knowledge" of the game that is being played. Introduced by John C. Harsanyi in 1967, rather than players knowing every parameter of the game situation such as utility functions, possible strategies, and information held by other players, each player knows a probability distribution from which these will be drawn. In his paper, Harsanyi says that this type of game can be thought of as a normal game, where "nature" goes first drawing from these probability distributions and assigning values before play begins without the players knowing which specific variation of the game they are playing \citep{Harsanyi1967}. Importantly, each player does know the probability distributions from which each value is selected. Formally, slightly modifying the definition given in \citet{Nisan2007}, Bayesian games are defined as follows,

\begin{dfn}[Bayesian Game]
	A game with (independent private values and) \textit{incomplete information} on a set of $n$ players consists of:
	\begin{enumerate}
		\item For each player $i$, a set of strategies $S_i$, letting $S = S_1 \times \ldots \times S_n$.
		\item For every player $i$, a set of types $T_i$, and a prior distribution $F_i$ on $T_i$. A value $t_i \in T_i$ is the private information that $i$ has, and $\F_i(t_i)$ is the a priori probability that $i$ gets type $t_i$. Letting $T = T_1 \times \ldots \times T_n$ and $F = F_1 \times \ldots \times F_n$.
		\item For every player $i$, a \textit{utility function} $u_i : T_i \times S \rightarrow \mathbb{R}$, where $u_i(t_i, s_1, \ldots s_n)$ is the utility achieved by player $i$, if their type is $t_i$, and the profile of strategies played by all players is $s_1, \ldots s_n$.
	\end{enumerate} 
\end{dfn} 

Using this definition to model our auction the types of players will consist of the publicly known distribution from which they are drawing their valuation. That is, our auction will consist of bidders who know their own valuation of the item being bidded on and the distribution from which each of the other players is drawing their own valuations. In a first-price, single item auction if player $i$ wins with bid $b_i$, we define the utility of the winner to be $u_i = v_i - b_i$, the difference between their valuation and their bid and the losers all get $u_i = 0$ since they did not receive the item. A strategy for a player is a function $s_i \in S_i$ that maps a valuation $v_i$ in support of $\F$ to a bid $s_i(v_i)$ (i.e.~taking into account the probability distribution for the other players). %TODO fix in support of

We move to the idea of equilibrium in this system. In games of complete information the central equilibrium concept is usually a Nash equilibrium, the set of strategies for all players in which each individual player cannot increase their utility by deviating from their strategy fixing the strategy of all the other players. That is, for every player if no one else changes strategy, their best option is to stay where they are, hence an equilibrium. This concept is now updated to give us a Bayes-Nash equilibrium where we must also incorporate the distributions for which players are drawing from. 

\begin{dfn}
	A strategy profile constitutes a {\em Bayes-Nash equilibrium} if for every player $i$ and every valuation $v_i$ that the player might have, the player chooses a bid $s_i(v_i)$ that maximizes her conditional expected utility where the expectation is over the valuations of the other players, conditioned on a bidder $i$'s valuation being $v_i$. 
\end{dfn}

\subsection{Formally Defining First-Price Auctions}
To analyze a first-price auction as a game, we give the notation we will be using. This notation comes from Roughgarden et.~al's 2017 survey of the subject of the price of anarchy in auctions. For a bid profile $\textbf{b} = (b_1, \ldots, b_n)$, we let 
\[
	x_i(\textbf{b}) =
	\begin{cases}
		1 & \text{if player $i$ is the winner} \\
		0 & \text{otherwise}.
	\end{cases}
\].

We let $p(\textbf{b}) = \max_{i \in \{1, \ldots, n\} } b_i$ denote the selling price. The utility that a player $i$ receives when their valuation is $v_i$ is 
$$u_i(\textbf{b}; v_i) = (v_i - b_i) \cdot x_i(\textbf{b})$$. 

We denote the {\em strategy profile}, or vector of strategies played by each player, by $\textbf{s} = (s_1, \ldots, s_n)$, where each $s_i$ is a function for player $i$'s valuation $v_i$ to their bid. We then let $\textbf{s}(\textbf{v})$ denote the strategy vector resulting from the vector of valuations $\textbf{v}$. For any given vector $\textbf{x}$, we use $\textbf{x}_{-i}$ to denote the vector $\textbf{x}$ with the $i$th element removed. Using this notation a first-price auction we can say that a strategy profile $\textbf{s} = (s_1, \ldots s_n)$ is a Bayes-Nash equilibrium if and only if 
$$ \mathbb{E}_{\textbf{v}_{-i}} [u_i(\textbf{s} (\textbf{v}); v_i) \ | \ v_i] \geq \EXP_{\valuations_{-i}} [u_i(b^{'}_i, \strategies_{-i}(\valuations_{-i}); v_i) \ | \ v_i] $$\citep{Roughgarden2017}.

\subsection{Example Auction}
We now turn to an example auction to clarify what has just been laid out. In this example we analyze the an auction between two players, Alice and Bob who are bidding on a candy bar where each select their valuations from the uniform distribution $[0,1]$. This is a first-price, sealed bid auction where they each submit a bid for the candy bar simultaneously. How are Bob and Alice supposed to decide what to bid on this auction? 

\begin{prop}
	In the first-price, sealed bid auction with valuation distributed on $[0,1]$, the unique Bayesian-Nash equilibrium is $\strategies = (s_1(v_1) = v_1 / 2, s_2(v_2) = v_2 / 2)$.
\end{prop}

\begin{proof}{\citep{Nisan2007}}
	First we show that this is a Bayesian-Nash equilibrium. Let us consider which bid $x$ is Alice's best response if Bob uses bidding strategy $s(v_2) = v_2/2$, where Alice's valuation is $v_1$ and Bob's is $v_2$. The utility of Alice if she wins is $v_1 - x$, and if she losses, $0$. Thus, her expected utility from a bid $x$ is $\EXP[u_1] = \Pr[\text{Alice wins with bid} x] \cdot (v_1 -x)$, where the probability is over $F_2$, the prior distribution of $v_2$. Now, Alice wins if $x \geq v_2/2$, and since $v_2$ is distributed uniformly in $[0,1]$ we can calculate the probability: $2x$ for $0 \leq x \leq 1/2$, 1 for $ x \geq 1/2$, and $0$ for $x \leq 0$. We see that the optimal value of $x$ is in range $0 \leq x \leq 1/2$ since $x = 1/2$ is better than any $x > 1/2$, and since any $x < 0$ will give utility $0$. Thus, to optimize the value of $x$, we find the maximum of the function $2x(v_1 - x)$ over the range $0 \leq x \leq 1/2$. Taking the derivative and setting this equal to zero, we get $2v_1 - 4x = 0$, which has solution $x = v_1/2$.
\end{proof}

Proving the uniqueness of this equilibrium requires the use of a fair bit of algebra and solving a differential equation. To see an example of proving the uniqueness see \citet{Levin2002}.

\subsection{Efficiency of First-Price Auctions}
Examples of the Bayes-Nash equilibrium have been solved for various combinations of the number of players and distributions from which they draw their valuations. With $n$ bidders it has been shown that the Bayes-Nash equilibrium strategy vector is composed of $s_i(v_i) = \frac{n-1}{n} v$ for all players $i$ \citep{Chawla2013}. Here the equilibrium is easy to calculate and efficient (meaning that the item will always be allocated to the player with the highest valuation). Neither efficiency nor ease of calculation are guaranteed for Bayes-Nash equilibria in this auction format. For example if we conduct an auction with two bidders, one choosing from the uniform distribution $[0,1]$ and the other from the uniform distribution $[0,2]$ it has been shown by \citet{Krishna2002} that the Bayes-Nash equilibrium for this auction is:
\begin{align*}
	&s_1(v_1) = \frac{4}{3 v_1} \left(1 - \sqrt{1 - \frac{3v_1^2}{4}}\right)\\
	&s_2(v_2) = \frac{4}{3 v_2} \left(\sqrt{1 + \frac{3v_2^2}{4}} - 1 \right)\\
\end{align*}

Here bidder one with the smaller valuation distribution knows that bidder two is more likely to have a higher valuation than them. Thus, player one must bid higher relative to their given valuation if they expect to win, and so they shade their bid less than bidder two. This can lead to bidder one drawing a lower valuation than bidder two, but outbidding them regardless and winning the item. This is inefficient. Moreover, it has been shown that solving many of these asymmetric Bayes-Nash equilibrium requires finding a solution to a system of partial differential equations many of which have no closed-form solution \citep{Roughgarden2017}. So even if we expect bidders to do their homework before an auction, they still might not know what to do. Given this, it is extremely hard to characterize or say things about what solutions to this format of auctions look like in general. However, just because we are not able to give a closed form for all of these equilibria, that does not mean we aren't able to characterize them in other ways.

\subsection{Price of Anarchy in First-Price Auctions}  
To try and get a sense of how inefficient first-price auctions can be, computer scientists have been applying a concept known as the {\em price of anarchy}, one way to characterize systems at equilibrium\footnote{Not that economists were uninterested in these questions of efficiency and welfare analysis in auctions before the computer scientists started studying them.}. The price of anarchy is a way to compare the social welfare of a system or a game at its best possible value to that of its worst possible equilibrium under strategic play. We must first define these concepts and then move to the point at hand. In the case of an auction, the social welfare is the sum of the utilities of the players plus the revenue of the auctioneer.

\begin{dfn}[Social Welfare]
	The \textit{social welfare} of a bid profile $\bidders$ when the valuation profile is $\valuations = (v_1, \ldots, v_n)$ is, 
	$$ SW(\bidders;\valuations) = \sum_{i=1}^{n} v_i \cdot x_i (\bidders).$$
	\label{dfn:SocialWelfare}
\end{dfn}

The price the winning bidder pays does not appear in this equation since the winning bidder is paying exactly as much as the auctioneer is getting, and this term cancels out. Hence, the social welfare is maximized when the bidder with the highest valuation wins the auction. If we let $x^*_i(\valuations)$ be an indicator variable for whether or not a player $i$ is the player with the highest valuation (ties broken arbitrarily), the maximum possible social welfare in a single-item auction is 
$$ \OPT(\valuations) = \sum_{i=1}^{n} v_i \cdot x^*_i (\bidders).$$

Now that we have mathematically described the social welfare in our system, we can define the price of anarchy. 

\begin{dfn}[Price of Anarchy]
	The \textit{price of anarchy} of an auction, with a valuation distribution $\mathcal{F}$, is the smallest value of the ratio:
	\begin{equation}
	\frac{\EXP_{\valuations} [SW(\strategies(\valuations);\valuations)]}{\EXP_{\valuations}[\OPT(\valuations)]},
	\label{eq:POA}
	\end{equation}
	ranging over all Bayes-Nash equilibrium $\strategies$ of the auction.
\end{dfn}


The above definition applies to individual auctions which are dependent on the choice  of $\F$~and~$n$. We generally only discuss the price of anarchy for the format of the auction which in our case is first-price single-item auctions.  The price of anarchy for the first-price auction format is then the worst possible price of anarchy for any choice of the number of players $n$ or valuation distributions $\mathcal{F}$. Note that the price of anarchy (for either an individual auction or the format of auction) is a number between $0$ and $1$, and that the closer it is to one, the "better" we can guarantee the system's social welfare will be\footnote{Much of the literature for POA (including the paper introducing the idea) defines it as the opposite ratio, Optimal/Worst-EQ where smaller values indicate better systems \citep{Koutsoupias1999}. For some reason the auction literature defines it as Worst-EQ/Optimal, so I will remain consistent with them. This is confusing as {\em price} of anarchy makes you think it should be a number that gets bigger as it gets worse.}.

Incredibly, bounds on the price of anarchy for the format of first-price auctions have been found. Again, this allows us to characterize how much worse the social welfare for the system could be at (Bayes-Nash) equilibrium no matter how many players we have or what distributions they are choosing their valuations from. This sort of guarantee is incredible, especially for systems where we may not want, or it may not be feasible to have a central authority pre-calculate the way to optimize social welfare in a system. Rather, we can trust that the system will perform at least so well under strategic interaction.

\begin{theorem}{\citep{Syrgkanis2013}}
	The price of anarchy in first-price single-item auctions format is at least $1-\frac{1}{e} \approx 0.63$
	\label{thm:POA}	
\end{theorem}

%TODO Find out if this is needed
%(Possibly include below thm, might be referenced by later proofs)
%\begin{theorem}
%	Every Bayes-Nash equilibrium of a first-price auction with correlated valuation distributions has expected social welfare at least $1 - \frac{1}{e}$ times the optimal welfare.
%\end{theorem}
%
%(CITE: Syrgkanis, 2014) (Tight bound for above)

Theorem \ref{thm:POA} tells us that no matter  how many players we have or what weird distributions we try and give them, we cannot construct a first-price auction that will achieve less than $0.63$\% of the optimal social welfare at Bayes-Nash equilibrium (if it exists). 
 
\section{Extending Results to No-Regret Agents}
These results hold for simple first-price auctions, but it is natural to ask questions about how robust these results are. First of all, how do people arrive at a Bayes-Nash equilibria if there doesn't exist a closed form way to express it? Secondly, do these results hold for mixed Bayes-Nash equilibria (randomizing between bidding strategies) or other larger, more realistic equilibrium concepts? To address these questions, Roughgarden et.~al use a set of extension theorems that take us through a general mechanism design setting and allow the class of equilibria our bounds hold for to be expanded. This extension will take us to a concept of no-regret learning. We briefly sketch key theorems from this thesis that lead us to an equilibrium concept that applies to learning agents.

\subsection{General Auction Mechanisms}

In order to understand (or even state) the proofs and theorems required to take us to our expanded set of equilibria, we must introduce  more notation and the idea of a general mechanism design setting and a general auction setting (this is unfortunate for us due to an expansion of scope, but actually makes these theorems quite powerful!). Mechanism design can be thought of as reverse game theory. Instead of thinking about how to play a game to maximize your utility, you are now thinking about how to design a game so that rational agents will behave in a certain way. This influence on players is generally exerted through payoffs and cost to play in the game. While we will not be spending much time with this math, it is useful to note that this framework reflects an ability for the auctioneer to fine tune how the game is played. In a general mechanism design setting, the auctioneer solicits an action $a_i$ from all of the players, $i$, from some action space $\mathcal{A} = \mathcal{A}_1 \times \cdots \times \mathcal{A}_n$. Given an action profile $\textbf{a} = (a_1, \dots, a_n) \in \mathcal{A}$, the auctioneer decides an outcome $o(\textbf{a})$ among the set of possible outcomes $\mathcal{O}$. This outcome includes a payment $p_i(o)$ that each player must give to the auctioneer. Denote the revenue of the auctioneer $\mathcal{R}(O) = \sum_i p_i(o)$. Players receive some utility as a function of their valuation, $v_i$ and the outcome which we write $u_i(o;v_i)$. Let $\mathcal{V} = \mathcal{V}_1 \times \cdots \times \mathcal{V}_n$. 

\subsection{Smooth Auctions}

Smooth auctions are a way of describing auctions in the general mechanism design framework that allow all possible deviations by players to be better accounted for in the math. The details of this are omitted from this paper, but for the complete details on this see \citet{Roughgarden2017}. The definition of a smooth auction is as follows, where $D^*_i$ is a new concept of an action distribution that is just a priori distribution over the belief about what actions the other players will take 

\begin{dfn}[Smooth Auctions]
	For parameters $\lambda \geq 0$ and $\mu \geq 1$, an auction is $(\lambda, \mu)$-smooth if for every valuation profile $\valuations \in \mathcal{V}$ there exist action distribution $D_1^{*}(\valuations), \ldots , D_n^*(\valuations)$ over $\mathcal{A}_1, \cdots, \mathcal{A}_n$ such that for every action profile \textbf{a},
	$$ \sum_i \EXP_{a_i^* \sim D_i^*(\valuations)}[u_i(a^*, \textbf{a}_{-i};v_i)] \geq \lambda \OPT(\valuations) - \mu \mathcal{R}(\textbf{a})$$
	\label{dfn:smoothauction}
\end{dfn}

 Now, following from the proof of theorem \ref{thm:POA}, Roughgarden et.~al show that first price, single payer auctions are $(1-\frac{1}{e}, 1)$-smooth. This is done by using the last step of their proof of theorem \ref{thm:POA}. They then go on to develop a series of extension theorems that allow us to characterize characterize the price of anarchy for equilibria of auctions when they are not just for pure-strategies. Specifically we will be looking at how these can be extended a set of outcomes for agents who learn to play the game as they go.

%\begin{theorem}(\citep{Roughgarden2017})
%	If an auction is $(\lambda, \mu)$-smooth, then for every profile $\mathcal{F}_1, \ldots, \mathcal{F}_n$ of independent valuation distributions over $\mathcal{V}_1, \ldots, \mathcal{V}_n$, every Bayes-Nash equilibrium of the auction has expected welfare at least $\frac{\lambda}{\mu} \cdot \EXP_{\valuations \sim \mathcal{F}}[\OPT(\valuations)]$.
%\end{theorem} 




\subsection{No-Regret Learning}
Consider an auction with $n$ players that is repeated for $T$ time steps. At each iteration $t$, bidder $i$ draws a valuation $v_i$ from $\mathcal{F}_i$ and chooses an action $a_i^t$ which can depend on the history of play. After each iteration, the players observes the actions taken by the other players. 
%(TODO: potentially can be relaxed to only need to see utility for their own action, which makes simulating easier. Need to follow up on source). 
A player $i$ is said to use a no-regret learning algorithm if, in hindsight their average regret (difference between average utility of strategy vs algorithm) for any alternative strategy $a_i^{'}$ goes to zero or becomes negative as $T \rightarrow \infty$. When all players use this algorithm it results in a vanishing regret sequence.

\begin{dfn}[Vanishing Regret]
	A sequence of action profiles $\actions^1 , \actions^2, \ldots, a^T$ is a \textit{vanishing regret sequence} if for every player $i$ and action $a_i^{'} \in \mathcal{A}_i$,
	$$ \lim\limits_{T \rightarrow \infty} \frac{1}{T} \sum_{t=1}^{T}(u_i(a_i^{'}, \actions_{-i}^{t};v_i) - u_i(\actions^t; v_i)) \leq 0$$
	\label{dfn:noregret} 
\end{dfn}

\begin{theorem}[\cite{Roughgarden2007}]
	If an auction is ($\lambda, \mu$)-smooth, then for every valuation profile $\valuations$, every vanishing regret sequence of the auction has expected welfare at least $\frac{\lambda}{\mu} \cdot \OPT(\valuations)$ as $T \rightarrow \infty$.  
\end{theorem}

The end result is that since single-payer first-price auctions are $(1-\frac{1}{e}, 1)$ smooth, we know that this bound will hold for auctions with no regret learning agents. Thus, we will try and build a simulation that uses one of the algorithms that fulfills this property and demonstrate that for any arbitrary number of players and distributions from which they pick their valuations from, they converge to an equilibrium social welfare greater than $1-\frac{1}{e}$.

\chapter{Simulating the Price of Anarchy}

This chapter constructs a novel simulation of first-price single-payer auctions to demonstrate that the ratio of actual social welfare to optimal social welfare is within the bounds given by the price of anarchy for this format. First, the construction of this simulation is discussed. Next, we demonstrate the results of running the simulation on bidders using known Bayes-Nash equilibrium strategies and an arbitrary bidding strategy. Finally, we demonstrate that these bounds also hold for no-regret learning agents in a fixed action variant of the first-price auction.

\section{A Simulated Auction Environment}

To simulate sequential first-price auctions, we construct a program in python that allows us to create an arbitrary number of bidders, each with valuation distributions of our choice who simultaneously bid on an item being auctioned for as many sequential auctions, or rounds, as we choose. Each round represents an auction for a new, but similar item where the valuation distributions for each bidder remains the same. At the beginning of each round, every bidder draws a new valuation from their distribution. The bidders then each simultaneously submit a bid to the auction and the winner is determined by the highest bid (ties are broken with equal probability among those with the same bid). After the winner is selected, they are given utility $u(v, b) = v - b$, the difference between their valuation and bid that round. At each round the total social welfare is $v_i$, the valuation of the winning bidder as per definition \ref{dfn:SocialWelfare}. The optimal social welfare each round then is the highest valuation of any bidder. These values are summed across auctions to get the total social welfare for this sequence of auctions and to see what the optimal social welfare would have been. This is laid out in the pseudo-code version of the sequential auction below where we use the super script $t$ to denote which round each variable is from\footnote{The "$\leftarrow$" symbol used in the algorithm means assignment of value. For example $x \leftarrow 1$ is the variable $x$ is assigned a value of $1$. This reduces the ambiguity of using the "$=$" symbol which could be also be a statement or proposition in pseudo-code.}.\\

\begin{algorithm}[H]
	Initialize $SW$, $OW$, and $POA$ to zero\\
	\For{$t = 1, \ldots, T$}
	{
		Each bidder draws their valuation $v^t_i$ from their distribution $F_i$\;
		Each bidder uses their strategy $s^t_i$ to submit a bid\;
		The highest bidder is assgined the object and they pay their bid, if tie, a winner is choosen randomly among them\;
		Each player has their utilites updated according to if they won the object\;
		\If{player $i$ wins the auction}{
			$SW \leftarrow SW + v^t_i$
		}
		\If{player $j$ has the highest valuation}{
			$OW \leftarrow OW + v^t_j$
		}
	
		$POA \leftarrow \frac{SW}{OW}$	
	}
\caption{Sequential First-Price Single-Item Auction}
\label{alg:main}
\end{algorithm}


\subsection{Simulating Bayes-Nash Equilibria}
Using the simulation outlined above, we first demonstrate that bidders using known Bayes-Nash equilibrium strategies have an average price of anarchy greater than $0.63$. First, we simulate the case of two bidders each drawing their valuation from the uniform distribution $[0,1]$. We are using the hard coded strategies that $s(v^t_i) = \frac{v^t_i}{2}$ for each bidder which was shown to be the unique Bayes-Nash Equilibrium in chapter 2. The results are shown in table \ref{table:1} below, where the cumulative price of anarchy is given up to the specified round specified 
\footnote{The POA is calculated as per algorithm \ref{alg:main}. %(TODO Fix manual ref).
}.

\begin{table}[h!]
\begin{center}
\begin{tabular}{ |c|c| }
	\hline
	Round & POA \\
	\hline
	1 & 1.0000 \\
	10 & 1.0000 \\
	100 & 1.0000 \\
	1,000 & 1.0000 \\
	10,000 & 1.0000 \\
	100,000 & 1.0000 \\
	\hline
\end{tabular}
\caption{Price of anarchy in two player symmetric auction}
\label{table:1}
\end{center} 
\end{table}

This example is obviously silly to simulate since this equilibrium is fully efficient. As each player bids half their valuation, the winner is always the player with the highest valuation. Hence the actual social welfare is always the same as the optimal social welfare: $$\frac{SW(\strategies(\valuations); \valuations)}{\OPT(\valuations)} = 1.$$

This does however give us a simple check of correctness for our simulation. We now move to simulate the more interesting example of two bidders with asymmetric distributions. We let bidder one chose their valuation from the uniform distribution over the interval $[0,1]$ and bidder two chooses their valuation from from the uniform distribution on the interval $[0,2]$. As stated in chapter 2, the unique Bayes-Nash equilibrium for this auction is:

\begin{align*}
&s_1(v_1) = \frac{4}{3 v_1} \left(1 - \sqrt{1 - \frac{3v_1^2}{4}}\right)\\
&s_2(v_2) = \frac{4}{3 v_2} \left(\sqrt{1 + \frac{3v_2^2}{4}} - 1 \right)\\
\end{align*}

Again, this auction is not fully efficient as the bidder who is drawing their valuation from the smaller distribution has to shade their bid less (bid higher relative to their valuation) in this equilibrium which can sometimes result in the person who values it less winning the item. We simulate this sequentially 100,000 times and get the following results as shown in table \ref{table:2}:
\begin{table}[h!]
	\begin{center}
		\begin{tabular}{ |c|c| }
			\hline
			Round & POA \\
			\hline
			1 & 1.0000 \\
			10 & 1.0000 \\
			1,000 & 0.9919 \\
			10,000 & 0.9924 \\
			100,000 & 0.9935 \\
			\hline
		\end{tabular}
		\caption{Price of anarchy in two player asymmetric auction}
		\label{table:2}
	\end{center} 
\end{table}

While this auction is not fully efficient, it is highly efficient. The price of anarchy in this auction never drops below 0.99. The main reason this happens is that while it is possible for the bidder drawing from the smaller distribution to win even if they have the smaller valuation, this only occurs when their two valuations are relatively close. This means that the social welfare lost in this case is not much, even if it is not fully efficient. In both of these auctions we see that they are well above the $0.63$ lower bound guaranteed for all first-price single-item auction formats. The fact that they perform so well is not surprising given that this is such a simple system. e would expect people who are rationally maximizing their own utility to mostly win if they have the highest bid. If anything, this simulation highlights the need to find an exact lower bound on the price of anarchy for Bayes-Nash equilibria as it would be more surprising if we could find an instance of this auction that produces highly inefficient outcomes.

\subsection{Minimal Intelligence Bidders}
Before going on to simulate the auction using no-regret bidders, we first move to establish a baseline for how well each auction setting (based on number of bidders and distribution choice) performs with agents that are not learning. The simulations of Bayes-Nash equilibria in the previous sections indicate that we can generally expect this system to perform well, but how well? How well will this system perform with even arbitrary bidders?
 
To try and answer this question, we construct agents that bid randomly between zero and their (current) valuation every round i.e. each players bid is chosen uniformly from the distribution $[0,v^t]$. We call these agents minimally intelligent since they are not overbidding, but they are also clearly using a nonsensical strategy for an auction. One should note that while this is a bad strategy, it is possible to formulate much worse strategies for our bidders from both utility maximization and efficiency standpoints\footnote{It's rather fun to think up such strategies! For example, if each bidder shaded little when they had a low draw and shaded a lot when they had a high valuation, this can lead to highly inefficient outcomes}. This in some way represents agents who are incapable of learning (or doing their homework before they arrive) and thus continually randomly select from all options. The results for simulating sequential first price auctions with 2, 10, and 100 agents using this strategy are shown in the table \ref{table:zero_int_symmetric} below.

\begin{table}[h!]
	\begin{center}
		\begin{tabular}{ |c|c|c|c| }
			\hline
			Round & 2 Agent POA & 10 Agent POA & 100 Agent POA \\
			\hline
			1 & 0.9000 & 0.8410 & 0.9859\\
			10 & 0.9312 & 0.8987 & 0.9419\\
			1,000 & 0.9279 & 0.8812 & 0.9480\\
			10,000 & 0.9150 & 0.8880 & 0.9467\\
			100,000 & 0.9164 & 0.8887 & 0.9470\\
			\hline
		\end{tabular}
		\caption{Price of anarchy in two player asymmetric auction}
		\label{table:zero_int_symmetric}
	\end{center} 
\end{table}

Table \ref{table:zero_int_symmetric} shows how the price of anarchy evolves through the rounds as the number of rounds goes to 1,000. Since the agents are not learning anything and the bids are randomly sampled each time, these efficiency results should approximately converge to the expected value as $T \rightarrow \infty$. To better illustrate the relationship between price of anarchy in each of these sequential auctions and the number of bidders, we graph the above simulation now run on 2 to 100 bidders. Each of these simulations is run 100 times, and the min, max, and average POA are graphed below in figure \ref{zi_symmetric}.

\begin{figure}[h!]
	\centering
	\includegraphics[scale=.8]{Figures/zi_symmetric}
	\caption{POA for 2 to 100 symmetric bidders}
	\label{zi_symmetric}
\end{figure}
 

 We see three things, one is that when all bidders are drawing from the same uniform distribution the market is incredibly efficient regardless of how smart the bidders are when choosing their strategy. Even with arbitrary strategies this auction still performs remarkably well. The second thing to note is that in this setting, as the number of bidders increases the social welfare also increases. This makes sense as we would expect the probability of a reasonably high valuation winning to increase as there are  more valuations per round. The third thing we notice, and most surprising of all, is that the increase in the price of anarchy as the number of players increases is not monotonic. At lower level of bidders, the efficiency actually decreases as we add more bidders. This seems to be because the probability that one player outbids another with a significantly higher valuation increases as you add players when you only have a few to begin with. Still though, the probability of someone with a high valuation winning is quite high and we see the price of anarchy approach 1 as $n$ goes to infinity. This is exactly what we expect as if we had a near infinite number of bidders, one of them would almost certainly bid arbitrarily close to their own valuation. 

We now conduct a similar simulation but with asymmetric bidders. In this case, half of the bidders are drawing uniformly from $[0,1]$ and half from $[0,2]$. The results are shown in table \ref{table:zero_int_asymmetric}. 

\begin{table}[h!]
	\begin{center}
		\begin{tabular}{ |c|c|c|c| }
			\hline
			Round & 2 Agent POA & 10 Agent POA & 100 Agent POA \\
			\hline
			1 & 0.1.000 & 1.0000 & 0.9726\\
			10 & 0.9115 & 0.8178 & 0.9331\\
			1,000 & 0.9054 & 0.8658 & 0.9310\\
			10,000 & 0.9092 & 0.8630 & 0.9297\\
			100,000 & 0.9112 & 0.8640 & 0.9304\\
			\hline
		\end{tabular}
		\caption{Price of anarchy in two player asymmetric auction}
		\label{table:zero_int_asymmetric}
	\end{center} 
\end{table}

Again we see that the market is quite efficient in this case regardless of how smart the bidders are. We again also see that it seems to converge to optimal (i.e. to 1) as the number of bidders increases but non-monotonically. We simulate this for 2 to 100 bidders again 10 times at each number of bidders with $T = 1,000$, where on odd numbers we allow the extra bidder to have distribution $[0,1]$. This is shown is figure \ref{figure:zi_assymetric}.

\begin{figure}[h!]
	\centering
	\includegraphics[scale=0.8]{Figures/zi_asymmetric}
	\caption{POA for 2 to 100 asymmetric bidders}
	\label{figure:zi_assymetric}
\end{figure}

Note that the baseline price of anarchy for minimal-intelligence agent auctions changes as we change the distributions. While the format seems quite efficient and well above the price of anarchy bounds (which again are only guaranteed for equilibria, which this is not) even with agents following these minimally intelligent strategies, we have no guarantee that there is not some way to construct this so that these agents wouldn't do much worse. Given that it seems as the number of agents increases, the efficiency also seems to increase, one would expect that this would need to be done by picking more interesting distributions for the bidders to choose from and only having two bidders.

\section{Simulating No-Regret Bidders}
With the results from the Bayes-Nash equilibrium demonstrated and an efficiency baseline established, we move on to simulating the bounds of no-regret learning agents in first-price auctions. Again, this is the equilibria we expect auctions to converge to if each agent is using a no-regret learning algorithm. To use these algorithms we do however have to make a concession to the environment we are simulating: we must now simulate an auction where the bidders are only allowed a finite number of actions.

\subsection{Multiplicative Weights Algorithm}
The no-regret learning algorithm we will use to train our bidders is called the multiplicative weights algorithm. It has been shown to satisfy the no-regret property in papers such as \citet{Littlestone1994} or \citet{Freund1999}, but our implementation of it comes from \cite{Roughgarden2016}. If each of our bidders use the no-regret learning algorithm our auction should converge to an outcome greater than the $0.63$ price of anarchy bound from chapter 2. 

Before giving the algorithm, a few words are probably necessary to understand where it comes from and what it is trying to do. First, this algorithm is what is called an {\em online} algorithm. That is an algorithm that takes its inputs sequentially as it goes rather than getting all of its inputs up front. Next, this algorithm and many other algorithms for players learning in games are based around the player only having a fixed number of actions they can choose from. For each round in the game, the player gets outside advice from "experts" who recommend to the player what to do at each round and the player picks among them to decide what to do. For us, these experts are our strategies that will map a valuation to a bid. At each time step $t$, the player picks the action to play and then after that, some adversary picks the utilities to assign for each action that could have been taken. This is a stronger condition than we will need as the adversary in a first price auction is the cumulative action of the other players, where the highest bid determines which strategies (if any) the player could have taken and won. However, in the general case this algorithm has been shown to be no-regret in the face of an adversary directly picking the utilities the learning agent receives (see \cite{Lugosi2006} for a broad overview of prediction with expert advice and learning in games).\\

\begin{algorithm}[H]
	Initialize $w^1(a) = 1$ for every $a\in A$\\
	\For{$t = 1, 2, \ldots, T$}{
		Use distribution $p^t = \frac{w^t}{\sum_{a \in A} w^t(a)}$ over actions to pick $a \in A$ and output $a$.\\
		Given the utility vector $u^t$, for every action $a \in A$ use the formula $w^{t+1}(a) = w^t(a) \cdot (1 - \eta u^t(a))$ to update its weight.\\
		\caption{Multiplicative Weights (MW) Algorithm}
	}
\end{algorithm}
\vspace{1cm}
The logic of this algorithm is simple. At each time step we see how well each of the possible actions performed and increase the weight, or probability of selecting that action in the future. This increase is done proportionally to how well the action did as determined and \textit{learning rate}, $\eta$, is chosen before starting the procedure. As demonstrated in \cite{Roughgarden2016} and \cite{Blum2007}, the MW algorithm is no-regret if $\eta = \sqrt{(\ln n) / T}$ where $n$ is the number of actions that this agent can choose from, and $T$ is the number of rounds that will be played (yes, this assumes that the player knows that up front).

This algorithm fulfills our purpose of simulating agents learning in auctions, but in some sense it is unsatisfying that the learning algorithm requires a finite set of actions that the bidder does not even get to choose. It would be more interesting if we were able to give our agents some reasonable algorithm that allowed them to formulate their own strategies or mappings between their valuation and bid rather than choosing from a pre-made set. This would introduce a whole host of other problems from the reasonableness of expecting agents to implement such algorithms to the ability to prove that such algorithms converge to an equilibria. Part of the beauty of using multiplicative weights is that it is simple and has nice mathematical properties. Using more interesting learning techniques these properties might no longer hold and then we would have to wonder what our simulation is really showing\footnote{This thesis grew out of an interest in doing just that, throwing "smarter" algorithms such as neural networks and machine learning into existing multi-agent simulations. It becomes hard to tell what the point of such simulations are when the dynamics might just be properties of the interaction of the specific algorithms used and not of the system itself.}.

\subsection{Uniform Distribution Simulations}
The first simulation we run is a repeat of the symmetric auction setting, but now using agents learning with the multiplicative weights algorithm. We create 101 strategies that bidders can choose from to shade their bid, from bidding zero percent of their valuation with one percent increases up to bidding 100 percent of their valuation, $S = \{ 0, 0.01 \cdot v_i, 0.02 \cdot v_i, \ldots, 0.99 \cdot v_i, v_i \}$. First, we run the simulation with two symmetric bidders each choosing their valuation from the uniform distribution over $[0,1]$. The results are shown below in table \ref{table:3}.

\begin{table}[h!]
	\begin{center}
		\begin{tabular}{ |c|c| }
			\hline
			Round & POA \\
			\hline
			1 & 1.0000 \\
			10 & 0.9517 \\
			1,000 & 0.9129 \\
			10,000 & 0.9578 \\
			100,000 & 0.9947 \\
			\hline
		\end{tabular}
		\caption{Price of anarchy in two player asymmetric auction with no-regret learning}
		\label{table:3}
	\end{center} 
\end{table}

Here we can see that after 100,000 rounds the price of anarchy converges to 0.99 and near perfect efficiency as the agents learn how to play the game. It's important to point out here that the above table is not an average, but simply one run of the simulation. Since each time we run the simulation it is possible for the bidders to learn to converge to a new equilibrium, the POA values for each simulation can be different. While we are mainly concerned with the lowest possible POA that the system converges too as a demonstration of the proven bounds, it is also interesting to see what the average and best case simulations resulted in, as these will give us more confidence about the robust efficiency of the system. We now repeat this using 2 through 100 agents, each symmetric and drawing from a uniform $[0,1]$ as above. We run 100 simulations with each number of agents, each for 1,000 rounds\footnote{This is a relatively small number, but necessarily chosen for the sake of computation time with 100 agents who learn.}. This gives us the following results as shown in figure \ref{figure:symmetric} 

\begin{figure}[h!]
	\centering
	\includegraphics[scale=.8]{Figures/symmetric}
	\caption{POA for 2 to 100 symmetric, no-regret bidders}
	\label{figure:symmetric}
\end{figure}

The efficiency in the minimum case with our learning agents is about as good as the best-case with the random guessing agents. It makes sense that the efficiency would increase as we would expect people with higher valuations to win more often as people learn to minimize their regret. It is theoretically possible in some games that social welfare could be better off with everyone choosing arbitrary strategies than under learning agents, but we see in this game, and with our learning agents, that is not the case.  

Finally, we simulate the case of having two bidders with asymmetric valuation distributions where one draws uniformly from $[0,1]$ and the other draws from $[0,2]$. We can see the result of simulating this 100,000 times in table \ref{table:5} below.

\begin{table}[h!]
	\begin{center}
		\begin{tabular}{ |c|c| }
			\hline
			Round & POA \\
			\hline
			1 & 1.0000 \\
			10 & 0.8901 \\
			1,000 & 0.9314 \\
			10,000 & 0.9770 \\
			100,000 & 0.9961 \\
			\hline
		\end{tabular}
		\caption{Price of anarchy in two player asymmetric auction with no-regret learning}
		\label{table:5}
	\end{center} 
\end{table}

Here we see the agents converge to a very good efficiency similar to the behavior we saw when using the Bayes-Nash equilibrium for this setting and better than in the two bidder minimum-intelligence case. Now again we conduct this simulation with 2 to 100 asymmetric bidders. For each number of bidders we simulate this 100 times and when there are an odd number of bidders there is an extra $[0,1]$ bidder. Each sequential auction consists of 100,000 rounds. The results are shown in figure \ref{figure:asymmetric}.

\begin{figure}[h!]
	\centering
	\includegraphics[scale=.8]{Figures/asymmetric}
	\caption{POA for 2 to 100 asymmetric, no-regret bidders}
	\label{figure:asymmetric}
\end{figure}

Figure \ref{figure:asymmetric} is similar to the figures we  have seen for every other auction. We see very high efficiency for two or three bidders, and then see a dramatic decrease as it goes to 9. We then see the numbers slowly climb back towards being fully efficient in the best, worst, and average cases. Comparing the results of our simulations, we see that while the asymmetric learners were more efficient for lower levels of bidders, they do not approach fully efficient in the same way symmetric no-regret bidders do as $n$ grows. Rather, it starts to level off faster. This seems reasonable based off of our expectations and similar results with minimally intelligent bidders.

\chapter*{Conclusion}
         \addcontentsline{toc}{chapter}{Conclusion}
	\chaptermark{Conclusion}
	\markboth{Conclusion}{Conclusion}
	\setcounter{chapter}{4}
	\setcounter{section}{0}
	
Our simulation demonstrated the efficiency greater than the price of anarchy bounds with no-regret bidders. The price of anarchy theorems that give the bounds for the efficiency of the system tell us that these should hold no matter what number of bidders and what distributions we give them. By all indications with both learning, and non-learning agents first-price single-item auctions are incredibly efficient. It makes one wonder how low one could get the simulated price of anarchy to be. It also begs for a better proof giving the exact lower bound on the price of anarchy for no-regret learners. Combining the results from the learning and non-learning agents we should feel incredibly confident in the efficiency of first-price auctions implemented in the real world. It seems like they behave remarkably well as a way to allocate resources to those who value them the most. 

Moving forward using simulations to understand the price of anarchy in auctions, there are two obvious ways to extend this research. One is to build a simulation for simultaneous first-price auctions or all-pay auctions (where everyone must pay regardless of winning) where the price of anarchy bounds for no-regret agents have also been proven using smooth-auctions (definition \ref{dfn:smoothauction}) \citep{Roughgarden2017}. For simultaneous auctions, or any sort of auction where bidders must make many decisions at once (in this case bids), simulating no-regret learning becomes much harder as the number of actions increases. The simulations presented here were already slow, its hard to imagine how long it would take to run a simulation of such a procedure. The other obvious extension is to try simulating this again using different algorithms. Specifically we could try different no-regret algorithms and see if they converge to similar efficiency outcomes, but it might also be interesting to try modern machine learning techniques. Again, if one uses techniques that don't allow us to bound the efficiency of their outcomes, it can be hard to know how sensitive the results are to particular parameters or what it might possibly do in every scenario. That is why no-regret learning is so nice as a stand in for humans. It is simple to implement, it is well behaved, and it is boundable. Hence, we should also expect human agents learning in auctions to make choices that lead to mostly efficient outcomes.


%If you feel it necessary to include an appendix, it goes here.
    \appendix
      \chapter{Simulation Code}
      The code for this simulation is available at the public GitHub repository:
      \begin{center}
      	\url{https://github.com/15rsirvin/Computational-Economics}
      \end{center}
      in the sequential auction folder. 
      There are different python files for each of the simulations where the relevant parts are changed to make them minimally intelligent bidders or asymmetric. The simulation is coded in python and mostly consists of loops to sequentially run each auction, update each bidder, and then allow each bidder to learn. We run each simulation in parallel on a different processor to try and speed up the process, but is is still quite slow for large numbers of simulations, rounds, or bidders. That is because this process is essentially $O(T \cdot n \cdot m)$, where $T$ is the number of rounds, $n$ the number of bidders, and $m$ is the number of simulations at that number of bidders. 
      
      We ran our code on a google cloud platform "compute engine" server to speed up the simulation process with a faster computer and to allow the simulation to run continuously. This enabled us to choose high values of $T, m,$ and $n$, despite how long the code takes to execute with such values.  
      
      If someone were to try and replicate parts of this thesis, I would recommend re-writing the code in a faster language like C or C++. Because this code scales linearly with any individual input, optimizing the speed at which the loop executes should have a massive effect on the total time it takes to compute. It might also be possible to algorithmically optimize this simulation so that it can scale sub-linearly. Ultimately we decided not to spend to much time optimizing the code as letting it run on a remote server was not a huge issue.
      

%This is where endnotes are supposed to go, if you have them.
%I have no idea how endnotes work with LaTeX.

  \backmatter % backmatter makes the index and bibliography appear properly in the t.o.c...

% if you're using bibtex, the next line forces every entry in the bibtex file to be included
% in your bibliography, regardless of whether or not you've cited it in the thesis.
    \nocite{*}

% Rename my bibliography to be called "Works Cited" and not "References" or ``Bibliography''
% \renewcommand{\bibname}{Works Cited}

%    \bibliographystyle{bsts/mla-good} % there are a variety of styles available; 
%  \bibliographystyle{plainnat}
% replace ``plainnat'' with the style of choice. You can refer to files in the bsts or APA 
% subfolder, e.g. 
 \bibliographystyle{APA/reedecon}  % or
 \bibliography{thesis}
 % Comment the above two lines and uncomment the next line to use biblatex-chicago.
 %\printbibliography[heading=bibintoc]

% Finally, an index would go here... but it is also optional.
\end{document}
